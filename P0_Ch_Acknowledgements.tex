%!TEX root = Thesis_David_Burns.tex

%Todo

% - write in your acknowledgements
% - read through it for continuity and completeness
% - check your writing style with that writing program
% - Get checked by mum
% - Get checked by Eliza

%Done


\chapter*{Acknowledgements}
\label{ch:ackn}
\addcontentsline{toc}{chapter}{\nameref{ch:ackn}}

people to thank

personal:
eliza
mum/dad
kim?

in group:
nassib
joel
rest of ilaqh

researchers:
zoran
martin
markus
konstantin
rest of physics and maths faculty



nassib:

I would like to extend my thanks to all the people who have lent their time and effort sup- porting the research presented in this thesis. I would particularly like to thank the following people.

Firstly, my supervisor Dr Graham Johnson, who has supported me through the whole hon- ours year. Though he is a busy man, he always made time to help with issues and impart advice and for that I’m grateful. I look forward to working with him in the future.

Secondly, the whole QIMR Berghofer Lung Bacteria Research Group at the Prince Charles Hospital, including Michelle Woods, Rebecca Stockwell, Dr Laura Sherrard, Dr Luke Knibbs and Prof Scott Bell, for graciously admitting me into the team and showing me the ropes. Without them this research could not have happened.
Thirdly, to my good friend and former honours student Joel Alroe, whose experience and generosity could always be sought upon without hesitation.

I would also like to thank all the faculty at ILAQH and QUT, whose guidance over the years have turned me into the researcher I am today, including Dr Rohan Jayaratne, Prof Lidia Morawska, Prof Zoran Ristovski, Assoc Prof Esa Jaatinen and Dr Konstantin Momot. I would also like to acknowledge all the members of ILAQH and the other honours students who created the positive environment that made the honours year so enjoyable.



joel:

I would like to firstly thank my supervisor, Prof. Zoran Ristovski, for his advice, training, troubleshooting, and support this year and throughout my undergraduate course. His enthusiasm for problem-solving and atmospheric physics is an inspiration.

Dr Branka Miljevic, Luke Cravigan, Marc Mallet and Anđelija Milič have each provided generous and invaluable guidance on the various instrumentation, analytical tools and background concepts surrounding aerosol research. Furthermore, without the time they devoted to performing field measurements and processing the resulting data, much of this thesis would not have been possible. In addition, I would like to thank Dr Konstantin Momot and A/Prof Esa Jaatinen for the years of support and genuine encouragement they have offered throughout my studies.

Finally, I would like to acknowledge both the larger ILAQH research team and my fellow undergraduate students for the incredibly positive and supportive communities they have each developed and without which, honours would have been a far less rewarding experience.