%!TEX root = Thesis_David_Burns.tex
\chapter{Conclusion}
\label{ch:conc}

%-----------------------------------------------------------------------------------------------------------------
%------------------------------------------------------------------------------------------------------------------
\section{Future Research}
\label{sec:futureres}

The next step is to use the \gls{ccam} modelling runs to run \gls{ctm} and \gls{glomap}. However, there are a number of precautions that need to be taken.

The analysis of current chemistry relating to \gls{dms} and its products in \cref{sec:chem} reveal a complex system. Ensuring that the chemical reactions are treated with respect to current theory should improve predictions. In \cref{subsec:postclaw} the different ways in which \gls{dms} is prevented from eventually forming \gls{ccn} was summarised. These mechanisms will need to be incorporated into the operation of both \gls{ctm} and \gls{glomapm}.

\gls{dms} surface flux values are required to provide input into \gls{ctm}. Changing ocean surface temperatures and wind speeds alter this flux through the mechanisms outlined in \cref{sec:dmssurf}. A surface flux model, potentially taking meteorological data from \gls{ccam} will need to be developed. This model should take into account wind speed, surface temperature and net surface radiation. It may be possible to alter the surface flux model, or change the setup for \gls{ccam}, to simulate a number of different scenarios resulting from changes in climate and changes in coral cover.

Once a complete run of the three models has been completed for October 2015, the modelling system should be cycled again for October 2016. Once the experimental campaign has completed and data has been extracted, the modelling data should be compared against the experimental data.

%-----------------------------------------------------------------------------------------------------------------
%------------------------------------------------------------------------------------------------------------------
\section{Closing Remarks}
\label{sec:closerem}

In this thesis the existing research surrounding \gls{dms}, and its effects on \gls{ccn} production, were examined. The focus was on modelling the system for the \gls{gbr} region. Investigating the atmosphere, its many layers, and the aerosols in it established the underlying theory for \gls{ccn}, and for atmospheric modelling. Reviewing the chemistry and role of \gls{dms}, and coral's production of it in the \gls{gbr}, identified the requirements for modelling chemical transport and the creation of new particles. The models to be used were examined for their function and viability. Finally the climatology of \gls{dms} was researched and methods for developing maps of \gls{dms} surface flux were found. The literature indicates that there is a necessity for localised models of the \gls{gbr}'s effect on cloud cover that existing research has not covered.

Modelling in \gls{hysplit} was performed to guide selection of experimental locations along the Queensland coastline. A modelling system was organised for regional modelling work in the \gls{gbr} and Queensland coast regions. The atmospheric part of the modelling system, \gls{ccam}, was run for the month of October, the month selected for the experimental campaign. The \gls{ccam} data was analysed and compared with measurement data from \gls{bom}.

The \gls{hysplit} work showed that the majority of the air arriving along the Queensland coast came from across the \gls{gbr}. A month, October, was chosen guided partly by the \gls{hysplit} work. Locations for data collection were also chosen along with a path and stopping points of a ship voyage.

The \gls{ccam} data mirrored the \gls{hysplit} work, showing that the prevailing winds were the trade winds providing \gls{gbr} sourced air to the coast for the majority of October. The data also indicated a low pressure system moving across the \gls{gbr} effecting surface temperatures and wind speeds across the \gls{gbr}. This highlights the importance of regional atmospheric modelling for modelling \gls{dms} production by the reef and its flux into the atmosphere.

The atmospheric conditions for the formation of clouds along the Queensland coastline are present in the \gls{ccam} model. The air in which these clouds are formed is almost certainly coming from across the \gls{gbr}. This fulfils some of the requirements for the \gls{gbr} influencing rainfall over the eastern part of Queensland. 

A comparison between \gls{bom} measurement data and the \gls{ccam} data was performed. The results showed an underestimation by \gls{ccam} of the range of temperatures experienced at the \gls{bom} stations. The maximum wind speed was also underestimated by \gls{ccam}. These results have been forwarded to the team working on the \gls{ccam} model.

What remains is whether \gls{dms} production is high enough, the chemical pathway from \gls{dms} to \gls{ccn} actually functions, the quantity of \gls{ccn} produced isn't swamped by other sources, and whether the pathway can occur fast enough before the air mass has moved over the coastline.