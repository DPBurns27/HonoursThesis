%!TEX root = Thesis_David_Burns.tex
\chapter{Introduction}
\label{ch:intro}

% make sure that every chapter that appears in here is given some love.
% you want to give the reader a foresight into what is to come.
% lay out the progression of ideas for them.
% your last paragraph here should be one line descriptions of each of the main sections to come

\section{Research Topic}
\label{sec:restop}

Climate change is a global issue effecting every country on Earth. The contributions to climate change are varied and complex requiring in depth study to provide as precise a picture as possible. Changes in the Earth's energy balance result from radiative forcing. Radiative forcing is changes in the amount of radiative energy absorbed or reflected by the ground and atmosphere. The radiative forcing component that currently has the largest uncertainty is aerosols (see \cref{fig:radforc}) \citep{intergovernmentalpanelonclimatechange:2015fa}. Aerosols are particles suspended in the air that can directly scatter or absorb radiation, or cause water vapour to condense onto them, acting as cloud condensation nuclei (\gls{ccn}). Clouds formed from \gls{ccn} reflect radiation back into space. As such the exploration of aerosols as a radiative forcing mechanism is a key area in understanding the larger issue of climate change.

\begin{figure}[!htb]
 	\centering
 	\includegraphics[width=0.8\textwidth,natwidth=1730,natheight=1248]{Fig/Literature_Review/Radiative_Forcing.png}
 	\caption{A diagram illustrating the various influences on radiative forcing along with their associated uncertainties. From the 2015 Intergovernmental Panel on Climate Change. The largest contributor to uncertainty is currently aerosols \citep{intergovernmentalpanelonclimatechange:2015fa}.}
 	\label{fig:radforc}
\end{figure}

A major cause of these uncertainties is the necessity for regionally specific aerosol knowledge \citep{intergovernmentalpanelonclimatechange:2015fa}. Aerosol composition and concentration differs greatly with changes in sources and atmospheric conditions. This regional variation translates to variation in direct scattering/absorption and cloud producing potential, leading to both local and global effects on climate. Thus it is important to develop tested, regionally specific models that take into account these variations \citep{cainey:2007jj, simpson:2014}. 

In 1987 the \gls{claw} hypothesis was defined in the seminal paper `Oceanic phytoplankton, atmospheric sulphur, cloud Albedo and climate'. The abbreviation \gls{claw} was taken from the initials of that paper's authors, Robert Charlson, James Lovelock, Meinrat Andreae and Stephen Warren. They proposed a feedback mechanism where stress driven marine biota produced chemicals that influenced cloud cover \citep{charlson:1987fw}. This paper generated a vast body of research involving many scientific disciplines. Dimethyl sulphide (\gls{dms}) is the core chemical responsible for the mechanism and is produced by phytoplankton, and as discovered more recently, coral \citep{raina:2013fj}. As the climate shifts towards increased temperature, regions like the Great Barrier Reef (\gls{gbr}) are increasingly losing coral coverage \citep{hoeghguldberg:1999bi}. It is therefore important to examine the potential effects on climate caused by \gls{dms} producing biota undergoing climate related reduction.

Initially, it is necessary to understand the role of the atmosphere and its constituents, and where aerosols and \gls{dms} are positioned within it. The pathways \gls{dms} proceeds down to form \gls{ccn} involve complicated chemistry \citep{barnes:2006ug} and must be explored to ensure the modelling mirrors current theory. The unique climatology of the \gls{gbr}, including the mechanism and scale with which coral contributes to \gls{dms}, needs to be established to provide localised inputs for the group of models. Modelling and the models themselves must be understood to ensure they are being applied correctly and to determine if they are sufficient for simulating the \gls{dms} to \gls{ccn} pathway. Finally, researching the method through which \gls{dms} enters the atmosphere, along with previous \gls{dms} to \gls{ccn} modelling attempts, provides insight into the modelling process and what areas of this research area remain unexplored.

Modelling \gls{dms} as it is produced, transformed and transported through the atmosphere, in the \gls{gbr} region, will provide needed insight into the mechanisms surrounding \gls{dms}. To do so requires a group of models simulating the different layers of the problem. The bottom most layer is \gls{csiro}'s Conformal-Cubic Atmospheric Model (\gls{ccam}) which provides information such as wind speed and temperature \citep{mcgregor:2005wz}. The middle layer is CSIRO's Chemical Transport Model (\gls{ctm}) which tracks chemical concentrations \citep{cope:2009tz}. The final layer is the Global Model of Aerosol Processes (\gls{glomap}) which simulates aerosol interactions and produces aerosol concentrations. 

This project takes the first steps in applying this trio of models, with a focus on the regional meteorological model \gls{ccam}. The domains centred on the \gls{gbr} and Queensland coastline were chosen and modelling runs were performed. The output was explored and tested against measurement data obtained from the Bureau of Meteorology (\gls{bom}), to establish the efficacy of the model. As this project was tied to an experimental campaign, back trajectory modelling was performed in the Hybrid Single Particle Lagrangian Integrated Trajectory Model (\gls{hysplit}) to find suitable locations for the experimental work.

\section{Research Gap}
\label{sec:resgap}

As the \gls{gbr} is such a large structure, and ocean temperatures are increasing \citep{hoeghguldberg:1999bi}, the effect these changes have on coral is critical. The production of \gls{dms} by coral is well established \citep{jones:2005ez,fischer2012atmospheric} and changes in coral coverage will effect this production. Furthermore, results from \citet{fischer2012atmospheric} indicate that while coral production of \gls{dmsp} increases in bleaching scenarios, the atmospheric \gls{dms} levels decrease drastically. \citet{fischer2012atmospheric} suggests this will decrease cloud cover due to the \gls{dms}, \gls{ccn} connection, further driving bleaching. The scale of current coral bleaching levels in the \gls{gbr}, increases the importance of exploring this relationship.

Global models are well developed for the relationship between \gls{dms} production and cloud coverage, however they contain large uncertainties \citep{woodhouse:2010ed}. \citet{cainey:2007jj} indicates that this is due to regional variability and calls for regionally specific modelling. While \citet{quinn:2011iv} used global modelling to refute the global negative feedback loop in the \gls{claw} hypothesis, they acknowledged that more regional modelling needs to be done to understand locally contained negative feedback loops. The satellite study performed by \citet{leahy:2013en} indicated the importance of including variation from local sources when modelling, particularly in regions where coral bleaching occurs.

It is clear that regionally specific atmospheric and aerosol models are necessary for reducing uncertainties in climate modelling, and that \gls{dms} producing biota serve a role in effecting climate. The \gls{gbr} is very high producer of \gls{dms} \citep{jones:2005ez}, with localised influences on production levels, making it an excellent candidate for regional modelling.


