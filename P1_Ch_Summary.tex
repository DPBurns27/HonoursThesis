%!TEX root = Thesis_David_Burns.tex
\chapter{Summary}
\label{ch:summ}

%--------------------------------------------------------------------------------------------------------------------------%

\section{Evaluation}
\label{sec:eval}

	The sources included in this literature review establish the underlying science of the \gls{dms} to \gls{ccn} process being analysed in this Honours project. They provide a clear image of the \gls{dms} pathways needed to create a model of this process in the \gls{gbr} and northern Queensland region. The papers chosen in \cref{sec:daande} illustrate the disputed state of the \gls{claw} hypothesis while care has been taken to understand all angles without prejudice. For \cref{ch:model}, foundational technical papers describing models were chosen to build a base understanding. Care was taken to include recent sources to ensure the current model versions are well understood. Modelling research on \gls{dms}, reviewed in \cref{ch:dmsclim}, was chosen for its relevance to the project and its impact on current literature.

%--------------------------------------------------------------------------------------------------------------------------%

% \section{Project Description}
% \label{sec:projdesc}

% 'project description'

%  	establish the parts of this process that you will be trying to model, probably just the DMS transmission and movement section, but really clarify it in terms of what youve been discussing in this section

%  	Initially, the location for future experimentation must be chosen. Using HYSPLIT the back trajectories for seven different locations along the Queensland coast will be analysed, averaging over a number of years for each month. This serves a dual purpose of finding the location and time for modelling in CTM and for the experimental site, and provides a foundation in scripting, modelling and data analysis. A method for visualising the output of the model will be devised such that monthly trends are made apparent. The site chosen will have the greatest probability of air arriving directly from the reef, with the lowest potential for introduction of anthropogenic aerosols.

% 	CTM will be obtained and explored using a simple test problem. Training may be required at CSIRO in Melbourne under Martin Close. The literature surrounding modelling NSS CCN, the GBR climate, the sulphur cycle and particularly flux parameters will be reviewed. Once the system being modelled is clear, it will be implemented in CTM and run for a variety of conditions and source/sink parameters. The data will be analysed and presented with particular focus on experimental reproducibility.


%--------------------------------------------------------------------------------------------------------------------------%

\section{Conclusion}
\label{sec:conc}

In this literature review the existing research surrounding \gls{dms}, and its effects on \gls{ccn} production, has been examined. The focus was on modelling the system for the \gls{gbr} region. Investigating the atmosphere, its many layers, and the aerosols in it established the underlying theory for \gls{ccn}, and for atmospheric modelling. Reviewing the chemistry and role of \gls{dms}, and coral's production of it in the \gls{gbr}, identified the requirements for modelling chemical transport and the creation of new particles. The models to be used were examined for their function and viability. Finally the climatology of \gls{dms} was researched and methods for developing maps of \gls{dms} surface flux were found. This literature review indicates that there is a necessity for localised models of the \gls{gbr}'s effect on cloud cover that existing research has not covered. The application of the \gls{ccam}, \gls{ctm}, \gls{glomapm} modelling system to the \gls{gbr} will elucidate the role of coral on climate and the impacts of coral bleaching.



		

	




	

